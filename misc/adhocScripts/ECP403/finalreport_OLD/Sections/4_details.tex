% =====================================================================================
This section details the design of R.F.I.D. Each functional block shown in Figure~\ref{fig:functional_diagram} is analyzed separately to give a comprehensive understanding of the project as a whole. The Power Supply, Time Keeper, Processing, and Local Storage blocks are each introduced and their associated circuits or algorithms are justified. See Appendix C for a complete parts list.

\subsection{Power Supply}

The Power Supply block of R.F.I.D. powers the whole project. This block's primary purpose is to supply 3.3VDC$\pm$10\% with at least 200mA of current, and no more than 200mVpp ripple to power the STM32l476RET microcontroller and SD card of the project. A buck converter steps the main battery voltage down from a nominal 3.7V to a nominal 3.3V. 
% TODO: The above paragraph needs ore editing. I'm moving on for now

The Power Supply block consists of a buck converter circuit that regulates the input voltage from a 3.7V nominal LiPo battery. The buck converter steps the battery's 3.7V voltage down to 3.3VDC$\pm$10\% for R.F.I.D.'s microcontroller and SD card. A buck converter was chosen instead of a linear regulator because switching regulators such as buck converters are typically more energy efficient. Energy efficiency is an important consideration for battery powered devices such as R.F.I.D., since an energy efficient device will require less frequent recharging or battery replacement. The buck converter circuit is shown in Figure~\ref{fig:buck_conv_schem}.

\begin{figure}[H]
    \centering
    \includegraphics[width=1\textwidth]{Figures/4_details/buck_schem.PNG} 
    \caption{Buck converter circuit schematic.}
    \label{fig:buck_conv_schem}
\end{figure}

The LTC3560 buck converter integrated circuit (IC) and its associated passive components regulate the LiPo battery voltage of 3.7VDC seen at the RUN pin to 3.3VDC at pin 1 of JP1. Either pins 1 and 2 or pins 1 and 3 of JP1 can be shorted to choose the path which the 3.3VDC signal follows. With pins 1 and 2 shorted, the 3.3VDC signal powers R.F.I.D. With pins 1 and 3 of JP1 shorted, the signal is diverted to H3 so that the power supply can be tested independently of the power demands of the rest of R.F.I.D. 

The buck converter IC needs to be able to meet the demands of the Power Supply block with regards to input voltage, output voltage, and output current. It needs an input voltage range spanning from at least 3V to 4.2V t to match the range of voltages which a LiPo battery could be charged to. It also needs to have an output voltage range which includes the specified 3.3V output voltage. The IC also needs to be capable of supplying at least 200mA of current to meet contract specifications. For these reasons, the LTC3560 was thus chosen as the power supply's buck converter IC. The LTC3560 accepts an input voltage range from 2.5V to 5.5V and can output between 0.6V and 5.5V. Both the voltage provided by the LiPo battery and the specified 3.3VDC$\pm$10\% output voltage fall within these ranges, which means that the IC meets the voltage requirements of the power supply. The LTC3560 can also supply a maximum current of 800mA, which exceeds the power supply's minimum specified output current of 200mA.

%The LTC3560 was chosen because it can meet the demands of the Power Supply block with regards to input voltage, output voltage, and output current. The LTC3560 accepts an input voltage range from 2.5V to 5.5V and can output between 0.6V and 5.5V. Both the voltage provided by the LiPo battery and the specified 3.3VDC$\pm$10\% output voltage fall within these ranges, which means that the IC meets the voltage requirements of the power supply. The LTC3560 can also supply a maximum current of 800mA, which exceeds the power supply's minimum specified output current of 200mA.

L1 was chosen based upon several consideration. This inductor's saturation current rating should be above the maximum expected DC output current of the LTC3560, as the inductor would be damaged if that current is exceeded during normal operation. The inductor should also have a low estimated series resistance (ESR), so as to minimize resistive losses, as the power dissipated by the inductor would be the the current flowing through it squared times its ESR. An inductor with a large ESR would significantly reduce the buck converter's efficiency.

Inductor L1 was selected based upon both the inductor ripple current that would be seen at the SW pin of the LTC3560 and the minimum saturation current that the inductor would need. The inductor ripple current is determined by
% Inductor ripple current

\begin{equation}
\label{eq:inductor_ripple_current}
    {\Delta}I_{L} = \frac{V_{OUT}}{f\cdot L}\left(1-\frac{V_{OUT}}{V_{IN}}\right) \cite{ltc:3560},
\end{equation}

where $V_{IN}$ and $V_{OUT}$ are respectively the input and output voltages of the buck converter, while f is the switching frequency (2.25MHz) at which the IC operates, and L is the inductor value. $V_{OUT}$ is set to 3.3V while $V_{IN}$ is set to 5V to design for the worst-case input voltage, as an optional 5V USB source was initially considered for the power supply. Equation (\ref{eq:inductor_ripple_current}) is thus simplified as

\begin{equation}
\label{eq:inductor_ripple_current_simplified}
    {\Delta}I_{L} = \frac{498.6n}{L}.
\end{equation}

The minimum saturation current required from the inductor is determined by

\begin{equation}
\label{eq:inductor_saturation_current}
    I_{L,sat.} = I_{L,max} + \frac{{\Delta}I_{L}}{2} \cite{ltc:3560},
\end{equation}

where $I_{L,max}$ is the maximum DC current which can be supplied by the LTC3560's SW pin (800mA) and ${\Delta}I_{L}$ is already known dependent on the inductor value. Equation (\ref{eq:inductor_ripple_current_simplified}) and (\ref{eq:inductor_saturation_current}) are then used to compare common inductor values and their associated inductor ripple currents and minimum saturation currents, as seen in Table~\ref{tab:inductor_value}.

\begin{table}[H]
\centering
\caption{Inductor saturation and ripple currents depending on L}
\label{tab:inductor_value}
\begin{tabular}{|c|c|c|}
\hline
L (${\mu}$H) & ${\Delta}$I_L~(mA) & $I_{L,sat}~(A)$ \\ \hline
1            & 498.7         & 1.049          \\ \hline
2            & 249.3         & 0.9247         \\ \hline
2.2          & 226.7         & 0.9133         \\ \hline
3            & 166.2         & 0.8831         \\ \hline
3.3          & 151.1         & 0.8756         \\ \hline
4.2          & 118.7         & 0.8594         \\ \hline
4.7          & 106.1         & 0.8530         \\ \hline
6.8          & 73.33         & 0.8367         \\ \hline
\end{tabular}
\end{table}

The chosen inductor must have a saturation current rating of at least $I_{L,sat}$ to avoid saturating the inductor core, which would reduce the inductor's inductance below its nominal rating. The values for L and their corresponding values for $I_{L,sat}$ from Table~\ref{tab:inductor_value}, as well as having a low ESR and large saturation current, were considered while choosing the inductor.

The MAMK2520T3R3M inductor was chosen for L1. Its inductance is 3.3$\mu$H$\pm$20\% while its saturation current rating is 1.8A, well above the minimum value of 0.8756mA for an inductance of 3.3$\mu$H from Table~\ref{tab:inductor_value}. The MAMK2520T3R3M also has a low maximum ESR of 156m$\ohm$ and a saturation current rating of 1.8A, which is more than twice the maximum current that the LTC3560 IC can source. The MAMK2520T3R3M is also shielded, meaning that the magnetic flux which it produces as current flows through it is more or less contained within the inductor and consequently less likely to interfere with nearby circuit components or parts.

Next, the input and output capacitors, respectively C10 and C11, were considered. The input capacitor serves two primary purposes. It impedes the change in voltage at the input of the buck converter, essentially smoothing the voltage signal. The input capacitor also provides a low impedance path to ground for alternating current (AC) noise. This AC noise is a result of the switching off and on of the MOSFET gate inside the buck converter IC that allows inductor L1 to store and discharge energy in the form of current from its magnetic field. This AC current associated with the buck converter's duty cycle is pulled through the input capacitor via the return path provided through ground from the output of the buck converter, as well as through the battery-provided unregulated power to the input of the buck converter.

The output capacitor, like the input capacitor, serves two primary purposes. The output capacitor stores energy to meet the instantaneous current demands of the load, while the MOSFET switch inside of the LTC3560 is in its off state and no current can flow from the buck converter's input to its output. The Power Supply block should have a fast response to increased demands from its load. Otherwise, important parts of R.F.I.D. such as the microcontroller or SD card could demand more current than could be quickly supplied. Even in the MOSFET switches' on state, the delay in the propagation of power through the buck converter circuit could be too slow to meet the demands of R.F.I.D. This capacitor also smooths the output voltage of the buck converter, reducing the output Power Supply block's voltage ripple.  

ESR and equivalent series inductance (ESL) are important parameters to consider when selecting both the input and output capacitors of a buck converter. The input and output capacitors ideally have zero ESR and ESL. The larger the ESR of the capacitor, the more power is dissipated as heat and thus the less efficient that the buck converter is. A larger ESR also increases the ripple voltage seen at both the input and output of the buck converter, as the voltage across a component is proportional to both the current flowing through the component and its resistance. The ESL of a capacitor provides a parasitic inductor that will store energy as the current flowing through it varies. The energy stored in this parasitic inductor is then released when the voltage across it, the voltage ripple seen at the output or input of the buck converter, is at its peak, causing voltage and current ripples. 

The maximum RMS current allowed by the input capacitor is also important, as the capacitor will be damaged if the RMS current flowing through the capacitor is above its RMS current rating. The maximum RMS current expected to be seen at the input of the buck converter is  

\begin{equation}
\label{eq:rms_max}
    I_{RMS,max} = \frac{I_{OUT,max}[V_{OUT}\left(V_{IN}-V_{OUT}\right)]^{1/2}}{V_{IN}} \cite{ltc:3560},
\end{equation}

where $V_{OUT}$ is the output voltage, $V_{IN}$ is the input voltage, and $I_{OUT,max}$ is the maximum output current expected to be sourced by the buck converter. Setting $V_{OUT}$ equal to 3.3V, $V_{IN}$ equal to 5V, and $I_{OUT,max}$ to 800mA results in an $I_{RMS,max}$ of 379mA. 

The output voltage ripple must, as required by contract, be less than 200mVpp. This voltage can be calculated via 
%The buck converter's output voltage ripple is determined by
\begin{equation}
\label{eq:output_voltage_ripple}
    {\Delta}V_{OUT} = {\Delta}I_{L}\left(ESR + \frac{1}{8\cdot f \cdot C_{OUT}} \right) \cite{ltc:3560},
\end{equation}

where the referenced ESR is that of the buck converter's output capacitor, and f is the buck converter's 2.25MHz switching frequency. Equation \ref{eq:output_voltage_ripple} was calculated for each capacitor considered. For simplicity, the input and output capacitors were chosen to be the same.

% Capacitor is wrong, double check, looks like actually using the 1276-1771-1-ND (CL31A226MQHNNNE)
Finally, the 22$\mu$F$\pm$20\% Samsung CL21A226MQCLRNC ceramic capacitor was chosen for C4 and C10. This capacitor has a low ESR of 9m$\ohm$ at 100kHz and an RMS current rating above 379mA. The effective capacitance is reduced by 50\% to 11$\mu{F}$ due to 3.3V bias voltage at the buck converter's output. After taking this into account, ${\Delta}V_{OUT}$ is calculated via \ref{eq:output_voltage_ripple} as 5.62mVpp. Considering other parasitic resistances, such as from PCB traces, would bring the output voltage ripple closer to a more likely value, but with a margin of allowable error greater than 150mVpp between the calculated and actual output voltage ripple, the 1276-1771-1-ND is a suitable selection for C10 and C4. 
% TODO: EPayne said she couldn't follow this parag. above...Look at later

%The capacitors C5, C6, C11, and C12, labelled with do not populate (DNP) in Figure~\ref{fig:buck_conv_schem} are extra capacitors that can be placed so as to reduce the resistance of the 

The last components to choose for the buck converter circuit are R5 and R6. The values of these resistors are selected to program, or set, the output voltage of the buck converter. The LTC3560 samples, or senses, the output voltage at any given time via its $V_{FB}$ pin and compares the output voltage with a 0.6V reference voltage, located within the IC, between the $V_{FB}$ pin and ground. An internal feedback loop inside of the IC uses the difference between these two voltages to determine the duty cycle of the IC's switching on and off, which is proportional to the the IC's step-down ratio. The step-down ratio is proportional to the buck converter's output voltage divided by its input voltage. Thus, the buck converter's output voltage can be chosen by selecting values for R5 and R6.

% within 2% according to the LTc3560's datasheet

The buck converter's output voltage is
\begin{equation}
\label{eq:DC_output_voltage}
    V_{OUT} = 0.6\left(1+\frac{R_{5}}{R_{6}}\right) \cite{ltc:3560},
\end{equation}

with the 0.6V constant voltage due to the mentioned reference voltage between the $V_{FB}$ pin and ground. Solving (\ref{eq:DC_output_voltage}) for R5 given an output voltage of 3.3V shows that R5 is required to be 4.5 times R6 for the output voltage to be 3.3V. Setting R5 equal to 900k$\ohm$ and R6 to 200k$\ohm$. satisfies this requirement. The 909k$\ohm\pm$0.5\% RC0805FR-07909KL resistor and the 200k$\ohm\pm$1\% AC0805FR-07200KL resistor were then respectively chosen for R5 and R6.

% mention that 909k was chosen because very few 900k resistors available commercially, must not be a standard value. Add the calculated output voltage (is a little different from 3.3V, still far within spec)

% =====================================================================================

\subsection{Time Keeper}
\label{ss:timekeeper}

The Time Keeper block is responsible for recording the passage of time so that R.F.I.D. can write a time and date stamp for each RFID tag that it logs on the project's SD card. These time and date stamps are required by the project contract. The time stamp should be human-readable, represented in terms of hours, minutes, and seconds, instead of only elapsed seconds. Ideally, the solution for recording time and date stamps is low powered and inexpensive. R.F.I.D. accomplishes these criteria by using the hardware real-time clock (RTC) that is included as a feature of the STM32L476RET microcontroller used for the project. The RTC feature is a significant reason for choosing this microcontroller for the project. 
%  overview
% topic sentence

%The real-time clock provides accurate time and date stamps at each point of data exchange.
% problems trying to solve?:
A real-time clock is a clock used by a computer or microcontroller to record the passage of time and dates in a human-readable format. In most digital watches, computers and similar devices, a real-time clock (RTC) is used to keep track of the passage of time. An RTC or similar timekeeping device is necessary if the device needs to perform scheduled tasks on time, or to provide accurate metadata.

% What are the criteria for a solution?
% hard, firm, soft realtime
Any time a real-time clock is in use, the system in which it is a component may be characterized as one of three cases; these are "hard" real-time, "firm" real-time, and "soft" real-time systems. They can be described loosely by the effect of a scheduled signal arriving late. A "hard" system would fail catastrophically, often with bodily harm as a result; for instance, retrograde thrusters on a moon lander. A "firm" system would have its output rendered useless, but could keep functioning; an example is a Morse code device. A "soft" system is mostly unaffected and can keep operating with a tolerable amount of error. R.F.I.D. is a "soft" system; data may still be collected and stored if the timing is slightly off. Consequences of this fact are outlined below.
\par


% accuracy of calibration
An RTC can be accurate in measuring elapsed time in measuring elapsed time, referencing the time of day in a human-readable format can be complicated. Even though some devices operate their clocks from individual power sources, interruptions may still occur. Solutions often involve polling the internet to double-check the stored time. However, because there is no degree of RTC accuracy specified for this project, and because device operation is not supported without power or in low power modes, a time and date may be provided at the time of programming. The simplest solution is to use the \_\_TIME\_\_ and \_\_DATE\_\_ constants, which are defined as per C standard by the pre-processor used by the host computer. These string constants are converted into the desired numerical format, after which the real-time clock may be appropriately calibrated and managed by use of an associated Hardware Abstraction Layer (HAL) library.
\par

The RTC for the STM32l476 microcontroller requires a low speed external (LSE) oscillator. This is a crystal oscillator which can be used by the microcontroller to provide a consistent frequency. This frequency can then be divided to lower frequencies, such as 1Hz to measure the passage of time on a second-by-second basis. The microcontroller requires a clock frequency of 32.768kHz. The crystal used, X1 in the complete schematic shown in Appendix B, is the ECS-.327-CDX-1128. This crystal provides the correct clock frequency of 32.768kHZ required for the microcontroller's RTC feature to work.

% clock drift handling
Over a period of extended use, a phenomenon called clock drift is common. The crystal may oscillate at a slightly different frequency than normally specified. This may be due to environmental factors or physical defects. This is a normal process and is in fact often used in computers to provide random number generator (RNG) seeds, or pseudo-random starting values. In cases such as these, or when an application is extremely time-sensitive, clock drift is often calculated with the use of a secondary RTC.

% =====================================================================================
\subsection{Processing}
This section describes the design and application of the software algorithms that R.F.I.D. uses. These include the control of input and output serial data, management of local storage via a FAT16 filesystem, and general operation. These algorithms relate to the operation of communication protocols to either receiving and transmitting data or the processing of that data.

%The software component pertaining to parsing output of the Time Keeper block was instead described in ~\ref{ss:timekeeper} to avoid confusion. 

\label{sss:FAT}
\subsubsection{FAT Filesystem} % How the data exchange actually happens

%NEEDS AN INTRO FOR SUBSUBSECTION HERE
R.F.I.D. needs to be able to be able to write formatted tag data with time and date stamps to a FAT formatted SD card. To accomplish this, it must be capable of communicating with an SD card via a digital communication protocol. R.F.I.D. must also be able to format data appropriately for the FAT filesystem present on the SD card. Otherwise, any written or logged data, as the project's final output, would not be readable. extensive software is required to conform to the requirements of writing to and reading from a FAT filesystem. There are also different variations of FAT filesystems that R.F.I.D. could support. 


% background info
%The File Allocation Table (FAT) is a type of filesystem where a table is used to describe the location 

%A File Allocation Table (FAT) filesystem  memory is segmented by clusters, which are contiguous blocks of memory.
%Files and directories (or "folders") are stored as directory entries that may occupy one or more clusters.
%Clusters may further be divided into sectors, which rarely hold any special significance. One exception is the boot sector, which contains vital information about the corresponding filesystem. Specifically, most of this useful information is contained within the BIOS Parameter Block (BPB), which is the first section of memory to be read by any device. 
%A
%\par
%to organize files contained in a data storage device, such as a USB drive or SD card. 

\label{ssss:FAT_background}
\textbf{Background of the FAT Filesystem:} A File Allocation Table (FAT) filesystem is a type of filesystem that uses a table, or index, called the FAT to mark the location of different chunks of memory. R.F.I.D. must be able to create files and log data on a FAT formatted SD card, without using an existing library. Consequently, a huge portion of the software developed for R.F.I.D. relates to adhering to a FAT standard. Thus, a description, or background, of FAT filesystems is required.

%Consequently, a huge portion of the software developed for R.F.I.D. relates to FAT16, the type of FAT filesystem that the project interfaces with. Thus, a description, or background, of FAT16 is required.
 
A FAT volume is divided into four regions. These regions are the reserved region, the FAT, the directory table, and the data region. Figure~\ref{fig:FAT_regions} shows, from top to bottom, the first memory region to the last.

\begin{figure}[H]
    \centering
    \includegraphics[width=0.5\textwidth]{Figures/4_details/fat_regions.png} 
    \caption{A FAT volume's four regions.}
    \label{fig:FAT_regions}
\end{figure}

Each region is further divided into sectors. All sectors on a FAT volume are the same size in bytes. The data region's sectors are also organized into clusters, with each cluster containing a fixed amount of sectors. The data region contains the contents of each file. Each file is allocated a number of clusters depending on its size. The directory table is where files are listed in separate entries. Each file entry contains information including the file's name, its size in bytes, and which cluster the file's data begins in. 

The FAT is a table of all clusters that are allocated for files. Each entry in the FAT has a value. This value may be 0x00 if a cluster is not yet allocated for a specific file, the number of the next cluster allocated for the file, or 0xFF if that cluster is the last one containing the file's data. The reserved region's first sector is the boot sector. The boot sector contains vital information about the filesystem that is needed to calculate the size and location of each region.

%Each cluster is represented by a 16-bit entry in the table, which is how FAT16 gets its name.

%An operating system will read the boot sector to determine how to traverse a FAT volume. It then will read the directory table to know which files are in the volume. To access a specific file for reading or writing, An operating system will go to the start cluster in the data region that is noted in the file's directory entry. If the operating system needs to access more data clusters, it will follow the allocated clusters for the file, like a linked list, by viewing the FAT. 


% different FAT versions
\textbf{Design Considerations:} FAT in fact accounts for a family of related filesystem standards, and so the necessity arises to choose which variant or variants to support. Options include FAT12, FAT16, and FAT32. Other derivations exist but are generally far more complex than needed for the scope of this project.
The chief difference between these three is the size of their cluster addresses. FAT12 uses a 12-bit address, FAT16 a 16-bit address and FAT32 a 32-bit address. A larger address entails the availability of more clusters, but also universally increased complexity.
%why?
\par

% size constraints
One key limiting factor to consider when developing this library is that of size. Each FAT variant has both a minimum and maximum available size. The minimum size for any variant is far less than even 1GB, and so can be neglected. The micro-SD cards most readily available to the design team are 4GB in nominal size, thus it's preferable to use a FAT variant that can work with a 4GB SD card without having to partition it. Table~\ref{table_fatsize} shows that either FAT16 or FAT32 would be sufficient, supporting a 4GB volume size~\cite{src_FAT-filesizes}.

%table of max volume sizes
\begin{table}[H]
\centering
\caption{Maximum volume sizes supported by FAT12, FAT16, FAT32.}

\resizebox{1\textwidth}{!}{%
\begin{tabular}{|l|l|l|}
\hline
\textbf{FAT variant} & \textbf{\begin{tabular}[c]{@{}l@{}}Approx. maximum volume size\\w/ 128 sectors/cluster\end{tabular}} & \textbf{\begin{tabular}[c]{@{}l@{}}Maximum volume size\\w/ 128 sectors/cluster\\ \\ (in bytes)\end{tabular}} \\ \hline
FAT12 & 255 MB & 267,694,024 \\ \hline
FAT16 & 4095 MB & 4,294,180,864 \\ \hline
FAT32 & 2047 GB & 2,198,754,099,200 \\ \hline
\end{tabular}%
}

\label{table_fatsize}
\end{table}

% \cite{src_FAT-filesizes}

The choice of FAT16 would meet the needs described, and is less difficult to implement than FAT32. As such, FAT16 was chosen as the sole supported FAT archetype. 

Features included in many FAT libraries are not required for the purposes of R.F.I.D. These include the ability to create and traverse sub directories, create files with `long' names, and handle file permissions. As such, the FAT16 library developed for R.F.I.D. features a basic set of core functions.

Figure~\ref{fig:FAT_layers} shows a model of the structure of a FAT filesystem and a corresponding library, modified in this case to include only the components essential to the operation of R.F.I.D.

\begin{figure}[H]
    \centering
    \includegraphics[width=1\textwidth]{Figures/4_details/FAT_layers.png} 
    \caption{Hierarchy of FAT library and associated data.}
    \label{fig:FAT_layers}
\end{figure}

\label{ssss:FAT_design}
\textbf{FAT16 Library Design:}
Three routines that make up the majority of the developed FAT16 library are used to navigate and modify the SD card's FAT16 volume. Each of these routines either initializes the files system, creates a file, or writes to a file. The initialization routine accepts a pointer to a struct that represents a FAT16 filesystem. It reads the boot sector on the SD Card to retrieve important values like the number of bytes per sector, number of sectors per cluster, and the total number of sectors on the volume The routine uses these and other parameters in the boot sector to calculate further information, such as each region's starting sector and size in sectors. These retrieved and calculated values are then stored in the filesystem struct for later use by the file creation and write routines.

The file creation routine initializes the contents of a struct that represents the file which data will be written to. It also initializes this abstracted file on the SD card's physical FAT16 volume. This routine reads the directory table's first sector into a buffer, which it then modifies to add an entry for the file to be created. This entry includes the name of the file, the size of the file in bytes, and the file's starting cluster. While there are additional attributes that a file entry may have, these three constitute the minimum required for a directory entry to be correctly read by an operating system. The file name is always "DATA.TXT," as R.F.I.D. only needs one file to write tag IDs and time data to. The start cluster is, for the same reason, always the same. The file size is set to zero, as a newly created file will begin with no data. The routine concludes its initialization of the file's directory entry by writing the modified buffer back to the FAT16 volume and storing the file attributes in the file entry struct so that the write routine can retrieve and update information it needs about the file.

The file creation routine then updates the filesystem's FAT to contain an entry for the file's starting cluster. This cluster is marked with 0xFF to signify that no further clusters are allocated for the file. The routine initializes a buffer for the FAT's first sector and writes it to the SD card. At this point, DATA.TXT can be viewed as an empty file by an operating system that supports FAT16.

The write routine receives a pointer to a buffer containing data to write to the file's allocated data clusters. It also receives pointers to the structs representing the filesystem and file to write to, as well as the amount of data, in bytes, to write to the file. This routine reads the file's end sector into a buffer that it appends data to so that the current sector, which may contain already written data, is not overwritten. The routine appends data in the buffer until all received data has been stored or the buffer is filled with one sector worth of data. At this point, the routine writes the modified buffer to the SD card. The routine will continue to iterate the current data cluster by one until all data has been written to the SD card or the first sector in the FAT has been completely filled with entries for the file, as this routine does not support a FAT that occupies more than one sector. Before the write routine exits, it writes FAT entries to allocate any new clusters required to contain the file's data. 





    % initialize the filesystem: 
    %  -read SD card's boot sector
    %  - put important values (bytes per sector, sectors per cluster, )
    % - calculate which sector each region starts in
    % - 
    % create the file:
    % - allocate buffers for two sectors, one in the directory table and another in the FAT
    % - set important file attributes in file struct, like start cluster, end cluster, file size (init to 0) file name
    % - Put values for the file's directory entry in the directory table, write the buffer to the SD Card
    % - mark the file's start cluster with 0xFF in FAT because the file is only allocated one cluster to begin 
    % - Write to the FAT buffer, write the FAT buffer to the first sector of the FAT
    
    % write to the file: (This one is a doozy)
    % - The write routine is called each time data needs to be written to a file
    % - Receives pointers to to the filesystem struct, the struct for the file to write to, the buffer containing data to write, and an integer for the number of bytes to write.
    
    % - Fills data into a sector-sized buffer until all the data to write is in the buffer or the buffer has has enough data to fill the current sector
    % - Writes the buffer to the file's current data sector
    % - iterates the current sector in file struct if the sector has been filled
    % - Also checks if the current cluster is filled, iterates the current cluster in file struct if so
    % - Continues this process until there are no more bytes left to write
    % - Allocates a new cluster in the FAT if a new cluster has been written to
    % updates the file size by the amount of bytes written before finally leaving the routine.


\label{ssss:FAT_application}
\textbf{Application of the FAT16 Library:} The operation of the library has so far been described without detailing the routines used to actually exchange data. These routines operate on a lower level, closer to the operation of physical hardware.  The abstraction of a FAT library must correspond to physical read/write operations. Figure~\ref{fig:FAT_readwrite} shows a brief overview of the four `steps' associated with any such data transaction. 

\begin{figure}[H]
    \centering
    \includegraphics[width=1\textwidth]{Figures/4_details/FAT-RWflow.png} 
    \caption{Order of occurrence for read/write routines.}
    \label{fig:FAT_readwrite}
\end{figure}

At the lowest level of digital logic, each data transaction is made via serial peripheral interface (SPI), a common serial communication standard. A board support package (BSP) is re-purposed from a commercial device similar in function to R.F.I.D., and coordinates these transactions. The BSP initializes and configures the microcontroller's SPI ports, handling of errors, and formatting of all data transfers. The BSP connects to the FAT library through the diskIO layer, which is essentially an intermediate step that segments data sector-by-sector and specifies the external mechanism by which to read and write data -- in this case, of course, the BSP. All diskIO functions are called by the basic read and write functions that reside in the highest level of the FAT library. These are the functions that will in turn be called directly by the main program.
% above paragraph needs some work

Whenever an electronic storage device, such as an SD card, is used, it is good practice to use an `eject' feature. This feature notifies the device, in this case the microcontroller, to finish writing to or reading from the SD card and to notify the user once the SD card may be removed. R.F.I.D. features a simple combination of an `eject' button and notification LED. An external interrupt is generated when the button is pressed, setting a flag that the user has opted to cease the logging of data and to eject the card. If a write routine is in progress, it will conclude and no more write or read routines will begin. Then, the notification LED is turned on to alert the user that the SD card can be safely removed without damaging it or corrupting its FAT16 filesystem.

% TODO
\subsubsection{General Microcontroller Operation}
% usart in uart out
\textbf{Choice of microcontroller:} R.F.I.D. uses the STM32l476RET6 microcontroller because it easily meets all design criteria. Most importantly, it features 512KB of flash memory, ample for the operation of a potentially large FAT library, and it is commonly used for similar FAT-based file storage applications. The device also has 64 pins, enough to support all peripherals in use with some moderate overhead for future modification. Lastly, the team was more familiar with the STM32 family of microcontroller than any alternatives.

\textbf{Communication:} The software component of R.F.I.D. is also responsible for interfacing with connected devices. The input device is described as an RFID tag reader in Section~\ref{sec:2_intro}. The presence of such a reader is not vital to the operation of R.F.I.D.; as such, it may be simulated. A Raspberry Pi 3B serves this purpose whenever possible. It supplies 24-character RFID tags that may be either randomly generated or defined in a separate list. These tags may be transmitted at random or defined intervals of time to mimic the operation of a functional RFID reader. This connection takes place over the Universal Asynchronous Receive/Transmit (UART) serial bus, as many RFID tag readers feature the same bus.

On the microcontroller end of the connection, data is accepted over UART via the Direct Memory Access (DMA) controller. DMA is simply a module included in the STM32 standard that allows safe transfer of information between devices in a way that does not interfere with program operation or shared resources. This feature is not strictly necessary for R.F.I.D. in its current state, but is worth including in the case that more devices were to be connected. 

The project also has a USART connection available so that useful debugging information can be obtained. A USART is similar to UART but may operate synchronously, meaning information can be sent and received at the same time. This bus displays information about the program status, success or failure of subroutines, and useful variables, to varying degrees of detail as specified by the user. The output would most commonly be a personal computer, because a USART-to-USB adapter simplifies the connection. For more rigorous debugging, the USART transaction may even be intercepted and interpreted by the host computer before being sent.

