This report describes the design, testing, and operation of the Robust Framework for Interfacing Devices (R.F.I.D.), a device which receives Radio-Frequency Identification (RFID) tag IDs from an external device and stores received tag IDs on an SD card for data storage. The device is essentially a data logger, meant to provide a way to record large volumes of scientific or medical data received from another device on an SD card. The device also logs the date and time at which tag IDs are received, storing them alongside each corresponding tag ID. R.F.I.D. stores all data on a File Allocation Table (FAT) formatted SD card which is inserted into the device. FAT is a common standard used for the organization of data on devices such as computers and SD cards, and was originally developed by Bill Gates for the Windows Operating System. A FAT file format for the SD card allows for the easy retrieval of recorded data by any user, as the SD card can be ejected from the device and recognized by most modern operating systems.

% Add a couple sentences for motivation and compare with a a similar device

%The motivation for this project comes from ...
%Compare with a similar device on the market...

The motivation for R.F.I.D. is to support radio frequency based indoor localization. Applications of indoor localization include the monitoring of home-bound individuals and asset tracking in industrial settings. Similar projects lean more toward the latter category. Two examples include Radiant RFID and Gao RFID, both of which provide the hardware and software infrastructure for the tracking and cataloging of personnel and assets. These projects vary greatly according to the application. R.F.I.D. is similar in scope, but is primarily concerned with logging RFID tag data. In this way, R.F.I.D. provides a base for a larger, more complex system.

The project contract includes a brief description of the project and lists the project specifications, and is included in Appendix~\ref{sec:A_project_contract}. The contract specifies that a DC-DC converter supplies 3.3V DC $\pm$ 10$\%$, with a maximum ripple of 200m$V_{pp}$ to the project, and that the converter be capable of supplying at least 200mA of current. The contract also dictates that the project must include a custom designed printed circuit board (PCB), that it must be capable of creating and writing files on a FAT file formatted SD card, that received RFID tag IDs must be stored with corresponding time and date stamps to the SD card, and that all file handling must be done without the use of a preexisting FAT file handling library.

The remaining sections of the report are outlined as follows. Section 2, provides a broad overview of the project. Section 3 provides specific details about the hardware and software developed for the project. Section 4 includes the results of the project. Section 5 concludes the report, while Section 6 includes the bibliography of outside sources used for the project and report.

%% Dan: So I've edited this section with what E Payne wanted. I am not sure if it's still subjunctive tense? 
% TODO: Still need to add motivation for project and also a similar device to compare/contrast. 