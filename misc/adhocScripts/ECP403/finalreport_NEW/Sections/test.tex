R.F.I.D., a device that receives radio-frequency identification (RFID) tag data through serial communication from an external device and stores the data on an SD card, has been designed and tested. A microcontroller \DIFdelbegin \DIFdel{received }\DIFdelend \DIFaddbegin \DIFadd{receives }\DIFaddend tag data from the external device through serial communication. Software features of R.F.I.D. included code to schedule inter-device data transfer and uploads, the design of a FAT filesystem library \DIFdelbegin \DIFdel{library }\DIFdelend for use with the onboard SD card, and serial communication between the microcontroller and peripheral communication device. Hardware features \DIFdelbegin \DIFdel{included }\DIFdelend \DIFaddbegin \DIFadd{include }\DIFaddend the interfacing of each module by use of a printed circuit board (PCB) and DC-DC conversion via a buck converter. The DC-DC converter's output voltage was measured as 3.33VDC with with a 56mVpp ripple while supplying a current of 271mA to a load. R.F.I.D. has been demonstrated to be capable of creating and writing a file on a FAT16 formatted SD Card, with received RFID tag data, date\DIFaddbegin \DIFadd{, }\DIFaddend and times written to the created file. \DIFdelbegin \DIFdel{All }\DIFdelend \DIFaddbegin \DIFadd{No }\DIFaddend file handling performed by the project \DIFdelbegin \DIFdel{did not use }\DIFdelend \DIFaddbegin \DIFadd{uses }\DIFaddend an existing FAT filesystem library. All project specifications were met or exceeded.



%All project specifications were met or exceeded.
% add results of project, and make abstract past tense


%This report describes the design, testing, and results of R.F.I.D., a system that receives radio-frequency identification (RFID) tag data through serial communication from an external device, and stores the data on an onboard SD card using a custom FAT filesystem.A microcontroller receives tag data from the external device through serial communication. Software features of R.F.I.D. include code to schedule inter-device data transfer and uploads, the design of a FAT filesystem library for use with the onboard SD card, and serial communication between the microcontroller and the peripheral communication device. Hardware features include the interfacing of each module by use of a printed circuit board (PCB) and DC-DC conversion via a buck converter. 